\documentclass{report}

\usepackage{import}
\import{../}{gov-style}
\addbibresource{../thesis.bib}

\begin{document}
\section{A grand theory of intelligence}

% some positive stuff here

As the rest of the conclusion presents a very optimistic take on intelligence generally, it is important to acknowledge that the scandals surrounding national intelligence agencies are well-documented, and in no way do I seek to excuse them. This thesis does not even touch the murky world of covert action that is most often associated with the CIA, including the attempted assassination of Fidel Castro, the Guatemalan \emph{coup d'\'etat}, and the Iran Contra affair. These are not uniquely American sins either. Though a lot of information about the inner workings of the KGB and the GRU is lost to history, modern Russian intelligence was implicated in an assassination attempt as recently as 2018. Illicit covert actions are disturbing elements of the intelligence apparatus which, morally, cannot be entirely separated from the pure information-gathering efforts that I have examined. Nonetheless, this thesis focuses on an element of American and Soviet intelligence that I believe to be a strong net positive for international security---the norm that states have a tacit right to verify intelligence through espionage, and that espionage countermeasures will be limited to certain narrow measures.

The previous two chapters examined situations in which technological advancements revolutionized intelligence, and the information that resulted was wisely put to use in a manner that reduced security tensions. It should go without saying that intelligence is not always used to reduce security tensions---it can just as easily have a net neutral, or even net negative effect. The most transformative intelligence failure of my lifetime was the 2003 invasion of Iraq, now considered, by virtually all ends of the political spectrum, to have been a colossal mistake. Much of the blame for that mistake rests at the feet of the American intelligence community, which the bipartisan Robb-Silberman Commission determined ``was dead wrong in almost all of its pre-war judgments about Iraq's weapons of mass destruction.''\footcite{commission_on_the_intelligence_capabilities_of_the_united_states_regarding_wmds_final_2005} The technological, organizational, or political changes that could have led the intelligence community to a different conclusion are a subject for a different thesis (or theses), but I invoke the Second Gulf War as a reminder that more trust in intelligence does not always lead to more secure outcomes, and my conclusion should not be interpreted as such.

I do believe, however, that preserving espionage as an intelligence-gathering practice leads to more secure outcomes in the international system. Therefore this conclusion will explain how espionage norms been adapted for modern times, and how they can best be preserved.

% Espionage gives security-seeking states a means to verify the intent and capabilities of their adversary without fundamentally altering that relationship. Yes, states go to great lengths to frustrate and embarrass each other on the international stage, but a foiled attempt at espionage rarely, if ever, changes the dynamics their relationship in such a way that would impact another aspect of their foreign policy. When both parties stay with the accepted bounds, they successfully cooperate to keep their intelligence clashes quarantined.

\subsection{The takeaways}
The contributions that I hope this thesis makes to the academic literature are threefold. First, I would like to situate peacetime espionage within the existing literature on international security. While the intersection of espionage and international relations has been studied before---beginning, in the modern context, as far back as 1962---its role in deescalating conflict is typically relegated to historical accounts of individual moments where Great People properly interpreted---or fatally miscalculated---the information that their intelligence operations provided them. My focus on Eisenhower and Khrushchev certainly does not resolve that gap, but in analyzing the application of espionage norms to a variety of technologies from the 1950s to the present day, I hope to start the conversation on how these norms operate independent of the people that promote them. The next section

Second, the strategic choices that ultimately made spy planes and spy satellites acceptable are ones that can be repeated. When the United States first introduced these novel spy technologies to the world, it made a deliberate effort to associate them with the espionage tradition. In the following sections, I will analyze why that worked, and what the success of these technological developments can tell us about how espionage interacts with the internet today. Certainly we see some obvious signs that espionage norms are at play is the cyber-realm when congresspeople quite literally say so, but now, armed with the power of history, we can be clear about what purpose those norms serve, and what kinds of militarization would test the limits of their tolerance.

And finally, I believe that espionage norms, both traditional ones and their technological offshoots, offer important lessons about the role of intelligence in securing the international system. The United States greatly benefits from a world in which global communication systems are considered off-limits to attack, and the space age should be seen as a phenomenal example of how that can be achieved. Intelligence norms helped marry security concerns---verifying arms control agreements---with the best interests of the entire world---that space be preserved as an arena where countries compete to out-innovate, rather than out-weaponize.

Intelligence serves a crucial de-escalatory purpose in the international system, but espionage norms are not magic, and the careful technological \emph{d\'etente} that years of space peace and Open Skies have accustomed us to should not taken for granted. The equilibrium is constantly under siege from rogue actors and revisionist powers, looking to gain an advantage wherever possible. In the final section, I will talk about some of the greatest threats to peaceful intelligence gathering today, and how the United States might preserve a ``peaceful uses of cyberspace'' policy from which, like the ``peaceful uses of outer space,'' it stands more to gain than any other.


% But I also think that the international community benefits in ways that even top American policymakers would be reluctant to acknowledge, because it would require admitting that successful espionage against the United States is a key part of that system.

\section{A theory on why espionage norms work}
% psychological bias towards more knowledge
% I believe that the desire on both sides to continue the practice of espionage writ large, even as one seeks to stop a particular instance of it, can be relatively easily explained. Every policymaker would like to their decisions to be as well-informed as possible. Without the benefit of hindsight, it is very difficult objectively gauge how much ``security'' is lost from a successful intelligence operation. By contrast, it is very easy for someone in charge of making an important decision to notice how much better off they are for having had access to some illicitly-obtained information. I don't mean to suggest that these trade-offs are always impossible to evaluate. Khrushchev was keenly aware of how the U-2 photography would affect his missile bluff. I just mean that in situations where it is difficult to evaluate relative gains, there is a strong psychological bias towards preserving your own view of the situation, even if it means allowing the adversary to have theirs as well.


\section{A theory on the applying espionage norms to new technologies}
\subsection{Old school espionage}
Espionage activities are considered part of fabric of behavior between states. While not explicitly condoned, civilian and military intelligence agencies routinely facilitate espionage---against both enemies and allies---that is often in violation of the domestic law of the nation being spied upon. Explicitly encouraging foreign nationals to undermine the security of their own state would seem to be a violation of domestic sovereignty, but it is for various reasons a normalized practice, and one that is occasionally even acknowledged by people in power. That delicate balance---where espionage is clearly illegal but also expected, acknowledged, and accepted---is maintained by a retaliation structure that carefully preserves every state's right to attempt espionage themselves.

This way this works is law enforcement agencies do everything possible to frustrate and discourage the act of espionage itself without discouraging the development of a market for the information that espionage provides. Employees and contractors that deal in high-level government secrets are carefully vetted. Counterintelligence officers engage in all kinds of expensive and time-consuming trickery to catch spies in the act. In most places, espionage is a capital crime, punishable by long prison sentences, torture, and in some cases, death. These measures all increase the price of intelligence, but they leave its buyers intact. As long as there are states willing to pay for confidential information, then there will always be someone desperate or disillusioned enough to provide it. States are not only aware of that fact, they rely on it, and limit the inter-state consequences for running a successful intelligence operation accordingly.

Espionage repercussions operate in an almost perfect game theoretic ``tit-for-tat'' model. The widespread reciprocity of espionage presents an ongoing series of opportunities to respond to the previous action, and the narrowness of traditional responses makes it almost impossible to impose a consequence that the the receiving state cannot impose in turn. When a particularly damaging intelligence operation is uncovered, the only countermeasure that typically impacts the country (rather than the agent) is expelling known intelligence officers from their diplomatic posts. Often the affected country will just respond by expelling some suspected intelligence officers as well, even though the expulsion that the state is responding too was justified by something that the state originally did. Eventually one side will start cooperating again, signaling that the reciprocal PNGing of diplomats---typically called, you guessed it, ``tit-for-tat'' expulsions---can stop, and both sides will return to a cooperative state, each preserving some of their HUMINT operations. At all times, it is clear that neither side can really gain a decisive advantage, and the punishment never ends in a mutual defection.

To someone outside the intelligence community, many of these specifics would not make a lot of sense, even though the general need to refrain from punishing espionage too harshly---because everyone is guilty of it, and wants to keep doing it---probably would. Most Americans would be surprised to learn that foreign embassies serve as intelligence outposts from which officers, in cover as diplomats, recruit potential agents within the American government---let alone that their own counterintelligence community is generally aware which diplomats are secretly intelligence officers, and allow them to remain operating until it is diplomatically convenient to send them home. The role of intelligence and its place within diplomatic protocol is an organ of international relations, which, like a biological organ, serves a clear purpose in the system of which it is a part, but whose design the end result of uncountable iterations and unknowable circumstances that cannot be fully explained today. For this thesis, it is enough to understand the rules that have come to define spooks and spycraft, and to understand how states cooperate to preserve them.

% The norms governing espionage as old as nations themselves. An entire chapter of \emph{The Art of War} is devoted to the use of spies. ``Hence it is only the enlightened ruler and the wise general who will use the highest intelligence of the army for purposes of spying,'' counsels Sun Tzu. Thucydides mentions the role of ``informants'' within the city of Syracuse, who fed information to their Athenian besiegers.
\subsection{New dog, old tricks}

Espionage conducted by human agents has a long historical tradition, but what happens when you replace humans with machines? Suddenly it is no longer clear that the old rules of intelligence can or should apply. This is the situation in which American policymakers often found themselves during the early years of the Cold War. The tense security environment and nuclear stakes drastically increased the value of good intelligence, and the government shifted its R\&D priorities accordingly. Technological advancements revolutionized how the US conducted espionage. But when the US deployed brand new, highly effective means of gathering technical intelligence, it was not a given that these would be treated with the same diplomatic minimization that HUMINT operations were. In other words, if operations were caught, that the international reaction would be contained.

The most significant problem facing these novel means of gathering intelligence is that the capabilities are no longer symmetrical in the easy way that dueling embassy operations are. That the main rational incentive for the

Why \emph{should} one country allow another to use its invincible camera plane, when that country is unlikely to develop its own invincible camera plane technology for many years? Certainly the leader of the country with the plane is going to argue that its use is legitimate under traditional rules of espionage---he's the one with the plane! The country being subjected to this new asymmetric capability is under no obligation to accept that premise.


The psychological bias that intelligence is less threatening than traditional military operations still existed, but it had to be deliberately associated with the new technological capabilities---and it had to supersede existing norms from other fields of IR.


In the context of the U-2 aircraft, for instance, that involved reconciling the existing role of planes in territorial sovereignty with the plane's new purpose as an intelligence collector.

When the U-2 was first developed, it was not at all clear which norm would supersede the other. As it happened, neither did---the interaction of existing norms was unique to both the craft itself and the way that it was marketed by the administration that invented it. There is no mathematical formula that can determine how an intelligence-gathering method is perceived. The U-2 is not 70\% spy and 30\% plane. Instead, it is a complicated intersection of the two. The U-2 is a craft that the Soviet Union---and frankly, any nation over which the US chooses to fly a plane without permission---had the undisputed right to shoot down, but also chose not to treat as an act of war, or even a particularly threatening act short of war. The U-2 was an ``outrage,'' but it was never seriously considered an act of war, despite Eisenhower's explicit concern that it would be.

% Though TECHINT played an important role in World War II, Dulles, the first Director of Central Intelligence was infamously biased towards traditional human spies. He took a long time to come around to understanding the substantial advantages the United States had to gain by investing in its TECHINT capabilities.


in routinely minimized by the party that was spied upon. When I say minimized, I mean that the response is of out of proportion with the harm done.

Just because a norm in favor of intelligence-gathering exists does not mean that a new technology automatically gets the benefit of that norm's protection.

In the case of both the U-2 plane and the Corona satellite, it was crucial to Eisenhower's strategy that they be perceived as civilian

In both of the technological moments examined here, it was the United States that most stood to gain by normalizing mutual use. It cannot be expected that this will always be the case. The deep dependency of American society on information systems today

\section{A theory on bringing intelligence norms in the cyberworld}
\subsection{The missing domain}
The United States military has traditionally recognize land, sea, air, and space as the four domains of conflict. Recently, it added a fifth: cyberspace.\footcite{carafano_americas_2018}

\end{document}

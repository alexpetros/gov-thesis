\begin{abstract}
Why do states allow each other to conduct espionage? This question has increased salience in the modern world, where The United States government repels uncountable attempts to access its systems every day. When a foreign adversary succeeds, however, the US rarely responds with punishment; it simply neutralizes the threat and works to prevent it from happening again. In this thesis, I theorize that United States cyber policy is guided by longstanding norms of international espionage, where the diplomatic repercussions are limited to a minimal set of consequences—--consequences intended to signal that the behavior is wrong without actually discouraging the initial attempt. I employ defensive realism, in which states use intelligence as a means to solve the security dilemma, to explain how these norms can persist among states of varying power. Then, I test past technological advancements that have also been used for intelligence---planes and satellites---and determine that they were built for the purpose of de-escalating conflict, and the United States successfully applied espionage norms to their use. United States cyber policy today could use these norms to secure the internet for peaceful use, and the US should act as a norm entrepreneur to ensure that it happens.
\end{abstract}

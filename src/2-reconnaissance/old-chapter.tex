\documentclass[12pt]{article}
\title{Chapter 2(?) --- The Character of Espionage}
\date{}
% \author{Alexander W. Petros}

% need it in .docx form? no problem!
% pandoc -s chapter-4.tex -o chapter-4.docx --bibliography="chapter-4.bib"; mv chapter-4.docx ~/Desktop

\usepackage{setspace}
\usepackage{graphicx}
\usepackage[margin=1in]{geometry}
\usepackage[english]{babel}
\usepackage[super]{nth}
\usepackage{csquotes}
\usepackage[authordate,
			noibid,
			backend=biber]
			{biblatex-chicago}

\usepackage[hidelinks,
			hyperfootnotes = false]
			{hyperref}

% text formatting
\doublespacing
\setlength{\parindent}{7ex}
\setlength{\parskip}{1em}
\AtBeginEnvironment{quote}{\singlespacing\small\interlinepenalty=10000}

\addbibresource{chapter-4.bib}

\begin{document}
\maketitle
\section{The norm against issue linkage}
Assume for a moment that you have been convinced that our policy response to cybersecurity incidents is guided by espionage norms. Why is that interesting?

Norms, by definition, do not impose hard constraints on a state's ability to act---they suggest the contours of what a state will consider appropriate. To understand the norms of a particular field is to understand the range of options on the table and the likelihood that any given one will be chosen. Ideally, those norms are robust enough that it is possible to extrapolate a sort of policy rubric, in this case explaining how one state assesses an intelligence threat and then makes a response. If we understand the criteria by which the threat is evaluated, then we can explain why the government appears to consider only a limited range of responses to major cyberattacks, and more importantly, predict which types of attack might elicit something stronger.

I propose that a single crucial practice underpins the post-Cold War international consensus on peacetime intelligence gathering: espionage is not explicitly permitted between states, but as long as certain boundaries are not crossed, the state is essentially guaranteed that retaliation will be limited to espionage-curtailing activities in the event that they are discovered. Even after having a particularly damaging intelligence operation uncovered---an agent deep within the adversary's government, for instance---the worst that is likely to happen is the expulsion of foreign nationals working the in embassy under diplomatic cover. Because the embassy personnel expelled are typically those that have been identified as possible intelligence agents, the move to expel diplomats is explicitly aimed at making espionage more difficult in the future. What it does not do is punish the offending state for having attempted the espionage in the first place.

Losing embassy staffers who were identified---likely correctly---as intelligence agents is a legitimate consequence. It guts the embassy's operational presence, which can have a lasting impact on that state's intelligence gathering ability in that nation for years to come.\footcite{macintyre_spy_2018} In terms of deterrence, however, diplomatic expulsions are the international relations equivalent of a parking ticket. When an embassy staffer is outed as a spy by the host nation, that staffer is still subject to the customary protection of diplomatic immunity. The worst that can happen to them is to be declared persona non-grata along with a handful of their colleagues and expelled from the country for ``activities incompatible with their diplomatic status.'' If the spy is not a diplomat, they will be tried and held captive (or killed), and if they are lucky they might get traded. Typically no further penalties are applied. There will not be economic sanctions, nor linkage that substantively impacts other diplomatic issues, only a variation in the number of diplomats expelled depending on how serious the breach.  Like with a parking ticket, you simply pay the fine when you get caught and you're free to drive again.

In this chapter, I will examine some of the most significant diplomatic incidents that resulted from intelligence-gathering operations, and assess the international fallout from each. With each case, I will demonstrate that despite ample provocation, the offended party's response was muted and explicitly limited to espionage-related diplomatic actions. In doing so, I make the case that if these incidents which received significant press did not merit a stronger response---or at least one that broadened the scope of the repercussions---then it is virtually impossible that harsh sanctions resulted from some lesser offense that escaped the public's attention. To be even more specific: if a great power state is caught spying on another in a way that both parties want to be permissible, the absolute worst that will happen in terms of material consequences is tit-for-tat diplomatic expulsions.

Occasionally, a state will overstep the mutually agreed-upon bounds, and in those cases the consequences are extended just enough to make it clear that this particular action is not within ``the rules the game.'' This still happens today, most recently with the Russian poisoning of traded former-spy Sergei Skripal.\footcite{masters_has_2018} In those case I will examine the consequences, and discuss how they are tailored to punish only the norm-violating action, not the espionage itself.

Of course, it is possible, even likely, that there are consequences to espionage exposure that are not made public. But it would be difficult to imagine that any such secret consequences have could have a significant deterrent effect.  The most severe repercussions are ones that are felt by the most people, which in turn puts pressure on the offending party that makes it less likely that their leaders are will choose to repeat the undesirable behavior. This is what Jonathan Karl is getting at when he asks Josh Earnst about why there were ``no diplomats expelled, no compounds shut down, no sanctions imposed,'' in the exchange quoted in Chapter 1. ``You don't do that stuff secretly.''\footcite{earnest_press_2017} The norm-breaking consequences, ones which would prove that states impose penalties for attempted espionage that are more significant than diplomatic expulsions, almost by definition have to be public.

% Add some stuff about deterrence theory here - requested book from the other library

The cases used to examine this premise are limited to Great Powers in the post-WWII era, for a few reasons. The first is simply scope. The purpose of these cases is to demonstrate that even in the worst instances, the only actions taken in response were limited to espionage-related retaliation. To broaden the time period would make it more difficult to argue that the cases chosen represent that maximal possibilities for diplomatic incident that arise from espionage. And since the Cold War is the historical period that immediately precedes our current one, it can't really be said that these examples are date.

Another scope concern is the countries examined. I use a pretty loose definition of Great Powers, at the very least including the United States, The Soviet Union, The United Kingdom, and France during the Cold War, adding China and Russia later on. While it is possible that some smaller intelligence services reacted to an incident in a norm-breaking way, they generally deferred to their NATO/Warsaw block superpower when it came time to respond. If an incident did not rise to the level of their concern, than it probably has no norm-defining power in this context.

Second, present-day espionage institutions have significant operational holdover from the Cold War, having been founded either shortly before (Britain's MI6, 1909), during (United States' CIA, 1947), or immediately after (Russia's FSB, 1995). Many of the intelligence techniques used today were developed in an environment where two superpowers were actively seeking new methods to undermine each other that, by design, fell short of conventional war. Espionage is one such method. The Truman doctrine is another. Though that geopolitical environment no longer exists, the norms that it spawned absolutely do.

\section{A timeline of espionage incidents}
How does one establish what the ``most significant'' intelligence-related incidents of the Cold War were? Crudely, they're the ones you've probably heard of. Julius and Ethel Rosenberg. Gary Powers in the downed U-2 spy plane. The Iran-Contra affair. The controversies which reached the press are the ones that would have applied the most pressure to leaders to respond in kind. In the interest of applying some scholarly rigor, however, the cases here will be drawn from a few established timelines, both official and academic.

Let's define an ``espionage incident'' as one which in which it was exposed that a person of import was working as an active agent of a foreign government, and that the intelligence they provided was used to inform the foreign government's actions. The presence of a human agent isn't strictly necessary, but in practice the reveal of a successful SIGINT (signals intelligence) or TECHINT (technical intelligence) operation almost never registers beyond the confines of intelligence agencies.\footnote{The exception that proves the rule here is when the Snowden leaks revealed in 2013 that the NSA was tapping German Chancellor Angela Merkel's phone. A counterfactual is impossible, but it's noteworthy that the incident was made public by a third party instead of the German intelligence agency. Rather than provoke incident, the typical reaction to discovering a SIGINT operation is simply to counteract it.} Actively abetting a foreign government by providing them government secrets is about the most serious an allegation possible that still remains entirely within the world of espionage. In the United States, this would land the perpetrator upwards of 20 years in prison. In the USSR they would likely end up dead.

This definition explicitly does not cover covert operations, which is a related field with entirely different aims. Covert operations attempt to sabotage or influence, intelligence operations attempts to inform. Though sometimes performed by the same agencies---the CIA infamously engineered the overthrow of democratically-elected Guatemalan President Jacobo Arbenez in 1954---they are treated by both academic literature and diplomatic practice as entirely different categories of offense.\footcite{fraser_architecture_2005} The Iran-Contra affair is not an espionage incident, by this definition.

[though I haven't finished writing about them, the timelines I'm using so far are from the CIA, a history of intelligence, and a book about failures in intelligence]


\section{}



\subsection{A distasteful but vital necessity}
On the morning of May 1, 1960, Soviet Air Defense forces detected a high-altitude aircraft flying over Soviet Tajikistan. Though they were were not yet able to conclusively identify it, the foreign aircraft was an American U-2 spy plane performing an reconnaissance overflight of the USSR---at a time when the Soviets were deeply embarrassed by their inability to prevent such flights.\footcite{orlov_u-2_2007} A previous U-2 overflight on April 9, less than a month prior, had lasted a full six hours and the aftermath caused an upheaval in the Soviet military. Leadership ordered an investigative commission to root out the shortcomings of the Air Force and Air Defense agencies, and a series of charts were drawn up anticipating the routes of future U-2 flights. The next time one came through, the Soviets were absolutely determined to shoot it down.

Though better prepared, the Soviets still proved unable to immediately stop the flight on May \nth{1}. A missile battalion in the plane's path was not on alert duty. Armed fighter aircraft were in the wrong positions. A frantic attempt to have another fighter plane literally ram it out of the sky was scrapped when the pilot failed to make visual contact.\footcite{orlov_u-2_2007} The highest possible levels of Soviet leadership were actively involved with the mission as it was taking place. A Soviet Colonel in the USSR Air Defense later recalled that Khrushchev ``clearly viewed the violation of their nation's skies by a foreign reconnaissance aircraft on the day of a Soviet national holiday, and just two weeks before a summit conference in Paris, as a political provocation.'' \footcite{orlov_u-2_2007}

The United States was aware how provocative these reconnaissance flights were. Though they provided incredibly valuable intelligence about the progress of the Soviet missile program, that intelligence came at the cost of incredible risk. The violation of a foreign adversary's airspace, even for the purpose of gathering intelligence, could be construed as an act of war. Starting in 1958, the American administration had ordered a drastic decrease in the number of overflights.\footcite[p.~51]{powers_operation_2004} Eisenhower was worried, correctly, about ``the terrible propaganda impact that would be occasioned if a reconnaissance plane were to fail.''\footcite[p.~162]{pedlow_cia_1998} But he was also under tremendous political pressure to learn more about the Soviet missile program, as right-wing senators hyped up fears of a so-called ``missile-gap,'' concerned that Soviet ballistic missile capabilities were advancing faster than our own.\footcite{licklider_missile_1970} With trepidation, he authorized two additional overflights, one for April 9 and one for later in the month.

A crucial factor emboldened Eisenhower and his team---most of whom, including DCI Allen Dulles, were much more enthusiastic about overflights than he was---to conduct that second mission. The USSR had been aware of the first flight, and the United States \emph{knew} that they were aware of it thanks to an onboard computer which had detected the Soviet tracking from the beginnings of the operation.\footcite{pedlow_cia_1998} That the US took Soviet silence on this issue as encouraging was a severe miscalculation. After years of bluster about the superiority of Soviet military forces, Khrushchev was not about to admit that his Air Defense could not not shoot an unarmed spy plane flying deep into his homeland. They couldn't do anything about it, and they refused to admit that by lodging a protest.\footcite[p.~59]{powers_operation_2004} This was a military shame, and it was going to be resolved with a missile.

Almost four hours into the spy plane's flight on May 6, pilot Francis Gary Powers felt a dull ``thump'' and ``a tremendous orange flash lit the cockpit and sky'' as a Soviet surface-to-air missile exploded behind him, ripping the wings off his plane and sending it into a tailspin.\footcite[p.~61]{powers_operation_2004} His subsequent crash and capture by the Soviet military became an immediate international incident. Three days after NASA quietly reported that they had lost a U-2 type weather reconnaissance plane, Khrushchev announced that the USSR had shot down an American plane which had flown into their airspace, which he called ``an aggressive provocation aimed at wrecking the Summit Conference.''\footcite[p.~112]{powers_operation_2004} Once Khrushchev gleefully debunked the resulting American cover story by revealing that they had taken Powers alive, Eisenhower came clean in a Press Conference where he memorably declared that espionage was a ``a distasteful but vital necessity'' to prevent another attack like Pearl Harbor.\footcite{eisenhower_news_1960}

Publicly, the USSR was absolutely not ready to accept espionage as a ``distasteful but vital necessity.'' The Paris summit was disastrous. Khrushchev launched into an angry rant on the first day and stormed out, immediately squashing any hope that pressing issues like the fate of West Berlin would be resolved there. He also canceled a planned visit by Eisenhower to the USSR that was scheduled to take place the next month. A contemporary newsreel about the conference said that ``in the course of two hours, Khrushchev brought US-Soviet relations to their lower point since the end of World War II.''\footcite{universal_studios_summit_1960} The shootdown had major diplomatic consequences that extended beyond the traditional realms of espionage. Khrushchev \emph{did} punish the attempt. He sent a signal that if the US wanted to be able to come to the table and do business with them, then they would have to stop the overflights.

\subsection{The silent shootdowns}
Despite the enormous media coverage, the U-2 incident actually demonstrates the beginning of and the limits to a developing norm about the permissibility of espionage, and was actually the last time that an act of pure surveillance caused a diplomatic incident between the Cold War Great Powers that extended beyond the immediate and necessary countermeasures. It is not quite enough to say ``the norm wasn't fully formed yet'' by way of explaining why the reaction was so harsh---it's important to understand exactly what about the U-2 overflight crossed the line, from the Soviet perspective. Doing so reveals how states define the limits of this norm by moving the consequences just outside the realm of espionage.

From the perspective of the United States, the overflights were a straightforward reconnaissance mission aimed at properly assessing the degree of threat that the Soviet missile program represented to their national security. The Soviets took it as a wartime operation violating their airspace and they responded accordingly, scrambling the jets and putting the SAM battalions on alert. ``As far as we were concerned,'' Khrushchev later wrote in his memoir, ``this sort of espionage was war --- war waged by other means.''\footcite[p.~446]{khrushchev_khrushchev_1974} In order to send a signal that \emph{this sort} of espionage was not permissible, Khrushchev explicitly rejected the diplomatic note of protest drafted by his foreign minister Andrei Gromyko. Espionage he could tolerate, the indignity of a deep penetration mission in Soviet airspace he could not.\footcite[p.~444]{khrushchev_khrushchev_1974}

We know that thinking on the reconnaissance flights shifted over the course of the Cold War, because despite the insistence of all parties to the contrary, the United States actually continued to use U-2 planes for reconnaissance missions!\footnote{The U-2 plane, a 60-year old aircraft equipped with a film camera, still flies missions today in Iraq and Afghanistan (\cite{phillips_u-2_2018}). The US has literally never stopped using it for intelligence purposes.} Officially, the CIA stopped sending missions into USSR airspace, and transitioned to he use of satellites for surveillance.\footcite{orlov_u-2_2007} Khrushchev claims that violations ceased after they shot down Powers, and credits himself with teaching the Americans that ``anyone who slapped us on our cheek would get his head kicked off.''\footcite[p.~449. An important note about Khrushchev's memoirs is that even though they were published after he was ousted, they were still written well before the end of the Cold War, smuggled to the West and published in part in 1970. While his writing is bombastic and of clear bias, it is certainly not pro-West, and there's no reason to suspect that he would be covering for the US here.]{khrushchev_khrushchev_1974} But the US continued to conduct overflights with the purpose of surveilling USSR military installations, and on at least two occasions the plan crossed over into Soviet territory. These violations of their territorial sovereignty were a far cry from the invasive surveillance of previous missions, and the Soviets responded with... a diplomatic protest. The US quickly apologized, and that was that.\footcite{orlov_u-2_2007} The US successfully normalized the use of U-2 planes for espionage purposes, within the confines of what the Soviets could tolerate, which was essentially that they weren't arrogant about it.

Gary Powers was neither the first not the last pilot to surveil the USSR, so why is he probably the only spy pilot you've ever heard of? Because he was the only one held up as an example. Two months to the day after Powers was shot down, July 1, 1960, a Soviet MiG-19 opened fire on an American RB-47H and captured the plane's navigator and the co-pilot. The lack of publicly-available documentation on this incident, especially given its proximity and similarities to one of the Cold War's most contentious moments, is shocking. While Powers was serving a 10-year sentence in Soviet prison, the two airmen were returned to the US with moderate fanfare shortly after John F. Kennedy's inauguration, in early February, 1961. The only relevant concessions the US made in return was to announce the discontinuation of its U-2 overflights (which Kennedy was already committed to) and to not make an issue of their illegal detention.\footcite{time_cold_1961}

To this day, the Air Force claims that the RB-47H was flying in international waters.\footcite{us_air_force_rb-47h_2015} Khrushchev claimed to have concrete proof to the contrary, but in his memoir essentially shrugs off this violation as the lone case where the US violated his new line in the sand.\footcite[p.~448]{khrushchev_khrushchev_1974} This is not event remotely true. Over the course of the Cold War at least 30 other American reconnaissance flights were lost without outrage or resolution, almost all of whom were collecting information on Soviet air defenses.\footcite{glenshaw_secret_2017} Many of them were so-called ``ferret flights,'' in which a repurposed bomber carrying radio equipment flies around attempting to gather information on the location of enemy radar stations. Their job was to deliberately get caught by Soviet radar, note that information, and then flee before their defenses were mobilized. Hundreds of airmen did not return. The fate of hundreds of this disappeared airmen was not resolved until 1992, when Russian president Boris Yeltsin and American president George H.W. Bush announced a joint commission to investigate these cases and provide resolution for the families. 126 remain unaccounted-for this day.\footcite{glenshaw_secret_2017}

Over the course of decades, the USSR did not make a diplomatic issue of airborn reconnaissance. They simply took the necessary immediate countermeasures, and shot down the planes that they could. And the United States, in turn, does not appear to have made an enormous fuss when they were shot down. There are a number of reasonable explanations for why the U-2 shootdown was blown up beyond the traditional scope for reconnaissance flights, which again resulted in the loss of hundreds of Americans over the course of a few decades, none of whom received the attention that Powers did. Partially the Soviets felt humiliated by a technology for which they had no countermeasures, and the uniquely insulting way the overflights flew deep into Soviet territory. Some in the Eisenhower administration believed that it was an excuse to justify a turn towards an intensification of hostilities that had already been decided by hard-liners in the Kremlin.\footcite[p.~328]{kistiakowsky_scientist_1976} Regardless of the varying percentages of truth these explanations might have, the u-2 incident was an effective execution of the USSR establishing a norm on the permissibility and limits of certain forms of espionage.

In 1992, members of NATO and The Warsaw Pact signed the ``Treaty on Open Skies,'' which establishes guidelines for unarmed aerial surveillance flights in the spirit of open information and de-escalation.\footcite{organization_for_security_and_co-operation_in_europe_treaty_1992} This concept was first suggested by Eisenhower in 1955.\footcite{center_for_arms_control_and_non-proliferation_fact_2017} In effect, it adds new elements to practice which had been going on illegally for years with little incident. With certain restrictions, the kind of flight that altered the course of the Cold War is now officially routine and permissible. And in February 1962, Gary Powers walked across a Bridge into West Berlin at the same time as Russian Colonel Rudolph Abel, an American prisoner. Powers' father had suggested to the State Department that they might be traded. A spy for a spy.\footcite[p.~239]{powers_operation_2004}


% After the main incident, it became a straightforward spy case





% Two problems: they're secret and it's not clear which norms are relevant.
% What vector are we looking to analyze?

% How does secrecy fit in?
% How does "operational shit" like no assasinations fit in

% \section{Espionage in International Law}
% What's missing with existing scholarship?


% Why is it neccesary to analyze espionage in the context of norms
% How does legal fit in?


\newpage
\printbibliography[heading=subbibliography]

\end{document}

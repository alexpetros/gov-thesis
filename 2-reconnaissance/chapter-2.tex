\documentclass[12pt]{article}
\title{Chapter 2 --- Aerial Reconnaissance}
% Inventing the spy plane
\date{}
\author{}

% need it in .docx form? no problem!
% pandoc -s chapter-2.tex -o chapter-2.docx --bibliography="chapter-2.bib"; mv chapter-2.docx ~/Desktop

\usepackage{setspace}
\usepackage{graphicx}
\usepackage{makecell}
\usepackage[margin=1in]{geometry}
\usepackage[english]{babel}
\usepackage[super]{nth}
\usepackage{csquotes}
\usepackage[authordate,
			noibid,
			backend=biber]
			{biblatex-chicago}

\usepackage[hidelinks,
			hyperfootnotes = false]
			{hyperref}

% text formatting
\doublespacing
\setlength{\parindent}{7ex}
\setlength{\parskip}{1em}
\AtBeginEnvironment{quote}{\singlespacing\small\interlinepenalty=10000}

\addbibresource{chapter-2.bib}

\begin{document}
\maketitle
% \section{The norm against issue linkage}
% Assume for a moment that you have been convinced that our policy response to cybersecurity incidents is guided by espionage norms. Why is that interesting?

% Norms, by definition, do not impose hard constraints on a state's ability to act---they suggest the contours of what a state will consider appropriate. To understand the norms of a particular field is to understand the range of options on the table and the likelihood that any given one will be chosen. Ideally, those norms are robust enough that it is possible to extrapolate a sort of policy rubric, in this case explaining how one state assesses an intelligence threat and then makes a response. If we understand the criteria by which the threat is evaluated, then we can explain why the government appears to consider only a limited range of responses to major cyberattacks, and more importantly, predict which types of attack might elicit something stronger.

% I propose that a single crucial practice underpins the post-Cold War international consensus on peacetime intelligence gathering: espionage is not explicitly permitted between states, but as long as certain boundaries are not crossed, the state is essentially guaranteed that retaliation will be limited to espionage-curtailing activities in the event that they are discovered. Even after having a particularly damaging intelligence operation uncovered---an agent deep within the adversary's government, for instance---the worst that is likely to happen is the expulsion of foreign nationals working the in embassy under diplomatic cover. Because the embassy personnel expelled are typically those that have been identified as possible intelligence agents, the move to expel diplomats is explicitly aimed at making espionage more difficult in the future. What it does not do is punish the offending state for having attempted the espionage in the first place.

% Losing embassy staffers who were identified---likely correctly---as intelligence agents is a legitimate consequence. It guts the embassy's operational presence, which can have a lasting impact on that state's intelligence gathering ability in that nation for years to come.\footcite{macintyre_spy_2018} In terms of deterrence, however, diplomatic expulsions are the international relations equivalent of a parking ticket. When an embassy staffer is outed as a spy by the host nation, that staffer is still subject to the customary protection of diplomatic immunity. The worst that can happen to them is to be declared persona non-grata along with a handful of their colleagues and expelled from the country for ``activities incompatible with their diplomatic status.'' If the spy is not a diplomat, they will be tried and held captive (or killed), and if they are lucky they might get traded. Typically no further penalties are applied. There will not be economic sanctions, nor linkage that substantively impacts other diplomatic issues, only a variation in the number of diplomats expelled depending on how serious the breach.  Like with a parking ticket, you simply pay the fine when you get caught and you're free to drive again.

% In this chapter, I will examine some of the most significant diplomatic incidents that resulted from intelligence-gathering operations, and assess the international fallout from each. With each case, I will demonstrate that despite ample provocation, the offended party's response was muted and explicitly limited to espionage-related diplomatic actions. In doing so, I make the case that if these incidents which received significant press did not merit a stronger response---or at least one that broadened the scope of the repercussions---then it is virtually impossible that harsh sanctions resulted from some lesser offense that escaped the public's attention. To be even more specific: if a great power state is caught spying on another in a way that both parties want to be permissible, the absolute worst that will happen in terms of material consequences is tit-for-tat diplomatic expulsions.

% Occasionally, a state will overstep the mutually agreed-upon bounds, and in those cases the consequences are extended just enough to make it clear that this particular action is not within ``the rules the game.'' This still happens today, most recently with the Russian poisoning of traded former-spy Sergei Skripal.\footcite{masters_has_2018} In those case I will examine the consequences, and discuss how they are tailored to punish only the norm-violating action, not the espionage itself.

% Of course, it is possible, even likely, that there are consequences to espionage exposure that are not made public. But it would be difficult to imagine that any such secret consequences have could have a significant deterrent effect.  The most severe repercussions are ones that are felt by the most people, which in turn puts pressure on the offending party that makes it less likely that their leaders are will choose to repeat the undesirable behavior. This is what Jonathan Karl is getting at when he asks Josh Earnst about why there were ``no diplomats expelled, no compounds shut down, no sanctions imposed,'' in the exchange quoted in Chapter 1. ``You don't do that stuff secretly.''\footcite{earnest_press_2017} The norm-breaking consequences, ones which would prove that states impose penalties for attempted espionage that are more significant than diplomatic expulsions, almost by definition have to be public.

% % Add some stuff about deterrence theory here - requested book from the other library

% The cases used to examine this premise are limited to Great Powers in the post-WWII era, for a few reasons. The first is simply scope. The purpose of these cases is to demonstrate that even in the worst instances, the only actions taken in response were limited to espionage-related retaliation. To broaden the time period would make it more difficult to argue that the cases chosen represent that maximal possibilities for diplomatic incident that arise from espionage. And since the Cold War is the historical period that immediately precedes our current one, it can't really be said that these examples are date.

% Another scope concern is the countries examined. I use a pretty loose definition of Great Powers, at the very least including the United States, The Soviet Union, The United Kingdom, and France during the Cold War, adding China and Russia later on. While it is possible that some smaller intelligence services reacted to an incident in a norm-breaking way, they generally deferred to their NATO/Warsaw block superpower when it came time to respond. If an incident did not rise to the level of their concern, than it probably has no norm-defining power in this context.

% Second, present-day espionage institutions have significant operational holdover from the Cold War, having been founded either shortly before (Britain's MI6, 1909), during (United States' CIA, 1947), or immediately after (Russia's FSB, 1995). Many of the intelligence techniques used today were developed in an environment where two superpowers were actively seeking new methods to undermine each other that, by design, fell short of conventional war. Espionage is one such method. The Truman doctrine is another. Though that geopolitical environment no longer exists, the norms that it spawned absolutely do.

% \section{A timeline of espionage incidents}
% How does one establish what the ``most significant'' intelligence-related incidents of the Cold War were? Crudely, they're the ones you've probably heard of. Julius and Ethel Rosenberg. Gary Powers in the downed U-2 spy plane. The Iran-Contra affair. The controversies which reached the press are the ones that would have applied the most pressure to leaders to respond in kind. In the interest of applying some scholarly rigor, however, the cases here will be drawn from a few established timelines, both official and academic.

% Let's define an ``espionage incident'' as one which in which it was exposed that a person of import was working as an active agent of a foreign government, and that the intelligence they provided was used to inform the foreign government's actions. The presence of a human agent isn't strictly necessary, but in practice the reveal of a successful SIGINT (signals intelligence) or TECHINT (technical intelligence) operation almost never registers beyond the confines of intelligence agencies.\footnote{The exception that proves the rule here is when the Snowden leaks revealed in 2013 that the NSA was tapping German Chancellor Angela Merkel's phone. A counterfactual is impossible, but it's noteworthy that the incident was made public by a third party instead of the German intelligence agency. Rather than provoke incident, the typical reaction to discovering a SIGINT operation is simply to counteract it.} Actively abetting a foreign government by providing them government secrets is about the most serious an allegation possible that still remains entirely within the world of espionage. In the United States, this would land the perpetrator upwards of 20 years in prison. In the USSR they would likely end up dead.

% This definition explicitly does not cover covert operations, which is a related field with entirely different aims. Covert operations attempt to sabotage or influence, intelligence operations attempts to inform. Though sometimes performed by the same agencies---the CIA infamously engineered the overthrow of democratically-elected Guatemalan President Jacobo Arbenez in 1954---they are treated by both academic literature and diplomatic practice as entirely different categories of offense.\footcite{fraser_architecture_2005} The Iran-Contra affair is not an espionage incident, by this definition.

% [though I haven't finished writing about them, the timelines I'm using so far are from the CIA, a history of intelligence, and a book about failures in intelligence]

\section*{Thesis so far}
\begin{em}
In this thesis, I seek to understand why nation-states (and the United States in particular) do not respond more harshly to to cyber-espionage. While the US takes the appropriate steps to prevent digital invasions, it appears that we they don't punish states diplomatically for hacking into our systems and attempting to steal state secrets. I theorize that this puzzling response to cyber-attacks can be explained by espionage norms that formed during the Cold War---norms that suggest a state will never extend the diplomatic consequences of an espionage attempt beyond the countermeasures necessary to prevent it. To make this connection, I will demonstrate the various forms in which this norm existed during the Cold War, then analyze how it influences our cybersecurity policy today.

To prove that this norm exists, each of the next three chapters will examine a different vector of espionage between the United States and the USSR: aerial reconnaissance, human intelligence, and other technological intelligence advancements like satellites. In each chapter, I will present a timeline of how that medium was used in the Cold War, investigate instances in which its use provoked the most severe diplomatic consequences, and argue that those consequences demonstrate an upper limit on the severity with which a state will respond to uncovering attempted espionage. In this chapter, we take a look at aerial reconnaissance.
\end{em}


\section{Introduction}


Planes have been used for spying for almost as long as they have existed at all. According to \emph{The New York Times}, the very first reconnaissance flight was in 1911, only eight years after Kitty Hawk, when ``Lieut. Piazzi today, for the first time in the history of warfare, made from this place an aerial reconnaissance against a hostile power'' surveilling the Turkish Infantry near Tripoli.\footcite{special_cable_to_the_new_york_times_air_1911} The ability to fly over an enemy's position to gain intelligence about their movements was so obviously valuable that countermeasures quickly followed. Early pilots famously attempted to shoot each other down with handguns by firing out of open cockpits in mid-air (emphasis on attempted), but by the time of the First World War, planes were already outfitted with synchronized machine guns and ground regiments equipped with anti-aircraft weaponry. Crucial information about military movements was going to come at a cost.

Though planes have long been spying, that didn't make them spy planes. Aerial reconnaissance first appeared in the press in 1911, but between then and 1960 the term ``spy plane'' appeared in the pages of \emph{The New York Times} exactly once.\footnote{According to my search of the archives. The instance mentioned here was in 1944, when the AP wrote that ``Germans tried to probe the secrets of the gathering Allied storm by E-boat patrol dashes [...] and by spy-plane coastal raids.''(\cite{the_associated_press_britons_1944}) In context of the article, though, it's clear that they are referring to military aerial reconnaissance, especially since it was written during the Second World War. This is a marked difference from a plane created for intelligence gathering purposes, which would come later.} That all changed in 1960, when the image of a spy plane crashed into the public consciousness as Gary Powers crawled out of a U-2 class aircraft over Soviet Russia. The downed craft was clearly not a generic fighter plane that had flown too close to the border, and it suddenly became necessary to categorize this new instrument, flying at unbelievable altitudes, whose sole purpose was to stay hidden and take pictures. The United States first tried to claim it was a ``weather craft.''\footcite{caruthers_soviet_1960} When that cover didn't hold up, a new story was born. The diplomatic fallout of the U-2 shootdown, which threatened to torpedo an upcoming NATO-Warsaw summit, had become the ``spy-plane case.''\footcite[This is the second time that ``spy plane'' as a term of art appeared in \emph{The New York Times}. There would be many more.]{reston_allies_1960}

Though it might sound like a distinction without a difference, creating the concept of a spy plane, a tool of peacetime intelligence rather than one of military reconnaissance, is a crucial moment in the history of espionage. That is not to say there is a hard boundary between military and non-military intelligence, either in purpose or collection. The overlap between military and civilian intelligence agencies---especially in the United States---is considerable. The Air Force was deeply involved with the U-2 Project, for example, even though it was technically under the jurisdiction of the CIA. The aerial photographs taken and the radio signals collected are used to inform all kinds of operations. To the extent that the various elements of American intelligence are able and willing to coordinate with each other, they often do.

What I do hope to demonstrate is that through the realm of aerial espionage, there is a concerted, intentional effort on the part of successive presidential administrations to distinguish in the eyes of our adversaries between the \emph{intent} of two types of intelligence gathering: espionage that is the prelude to a possible attack, and espionage that is intended to hopefully avoid one. This effort is messy and fraught with diplomatic pitfalls, but the implications are enormous. If there is some method of espionage---either the type of knowledge gained or the method used---that states agree not to punish each other for attempting, then it is possible that we may continue to allow it today.

\section{The Open Skies Treaty}
We know that Eisenhower wanted to make aerial surveillance permissible between states because he outright proposed it to the Soviet Union, as a means of reducing security tensions. In his prepared remarks at the Geneva Summit of 1955, Eisenhower laid out the following plan (emphasis mine):\footcite{eisenhower_president_1955}

\begin{quote}
\textbf{To give to each other a complete blueprint of our military establishments}, from beginning to end, from one end of our countries to the other, lay out the establishments and provide the blueprints to each other.
\newline

Next, to provide within our countries facilities for aerial photography to the other country---\textbf{we to provide you the facilities within our country, ample facilities for aerial reconnaissance}, where you can make all the pictures you choose and take them to your own country to study, you to provide exactly the same facilities for us and we to make these examinations, and by this step \textbf{to convince the world that we are providing as between ourselves against the possibility of great surprise attack}, thus lessening danger and relaxing tension.
\end{quote}

The UN-sponsored Geneva Summit, formally the \emph{Conference on the Peaceful Uses of Atomic Energy}, is a comparatively hopeful period in the Cold War, a conference promoting scientific diplomacy where such a cooperative information exchange might have seemed feasible.\footcite[p.~27]{luscher_nuclear_2018} Nonetheless, the scale of this offer is astounding. Eisenhower is  offering to let the the Soviets install literal military bases in the United States from which they can surveil our national defense infrastructure, so long as they allow us to do the same.\footnote{Also included were reciprocal nuclear facility inspections, an immediate nonstarter in the USSR, but also controversial within the American administration back home. (\cite{prados_review_2015})} In effect, his proposal is a formal agreement between two superpowers to endorse limited, state-sanctioned espionage for explicitly defensive purposes---exactly the norm I am proposing exists in practice, in a different form.

The offer was quickly rejected. In conversation over buffet that evening, Khrushchev countered, albeit politely, that the plan would not further disarmament at all because it would merely confirm the fragmentary information that their respective intelligence services already had.\footcite{department_of_state_memorandum_1955} That response is telling---though it is in no way an endorsement of the activity, it both acknowledges that espionage takes place between the two superpowers, and that it serves a potentially de-escalatory purpose.

Khrushchev also claimed that the offer was primarily a propaganda move on the part of the U.S., but backed down when Eisenhower essentially dared him to accept and call his bluff.\footcite[p.~2. This is a secondary source---the book cites a conversation by Eisenhower, quoted in NSC meeting memos that are stored in the Eisenhower library.]{bury_eisenhower_2014} The formal rejection came a few weeks later. ``While giving credit to this proposal's attempt to find a solution,'' U.S.S.R. Council of Ministers chair Nikolai Bulganin wrote in his report to the Supreme Soviet, ``\textelp{} aerial photography cannot yield the expected results, since both our countries contain limitless expanses in which, if one desires, everything can be hidden.''\footcite{bulganin_meetings_1955} It's a lukewarm excuse, but not a disdainful one. A bit earlier, he even said that the conference, as a whole, ``must be considered a definite success of the peace-loving forces.''

Was the offer genuine, or was it the stunt that Khrushchev claimed? In a book about the relationship between aerial reconnaissance and Eisenhower's New Look foreign policy, historian Helen Bury argues that a willingness to negotiate acceptable and enforceable agreements with foreign powers was central to his Cold War strategy.\footcite[p.~69]{bury_eisenhower_2014} More importantly, the intelligence that an Open Skies policy promised would have been invaluable in his ongoing effort to control defense spending, he was battled by deeply paranoid American institutions.\footcite[p.~212]{bury_eisenhower_2014} His commitment to Open Skies in practice, however, is still a point of some historical debate. In reviewing her argument, intelligence historian John Prados notes the contradiction between her portrayal of a president who sought arms control with Eisenhower's role in significant nuclear buildup. He also notes that Bury's evidence that Eisenhower kept pursuing the Open Skies concept after its rejection in 1955 is relatively circumstantial.\footcite[p.~233-234]{prados_review_2015}

Regardless of Eisenhower's intent, it is clear that the United States would have far more to gain from such a policy than the USSR, and it was within the best interests of the Soviets to reject the offer. Writing about the history of CIA's imaging program, former imagery analyst David Lindgren points out that Khrushchev had the advantage of huge amounts of publicly-available information about US military installations through newspapers, maps, and official aerial photographs, while Soviet restrictions on press freedom denied symmetrical data.\footcite[p.~38]{lindgren_trust_2000}Allowing for an Open Skies policy would also have revealed that Soviet air defenses were not nearly as far along as their leadership claimed. Khrushchev's son would later write that his father's overriding concern at the time was to conceal Soviet weakness.\footcite[p.~133]{brugioni_eyes_2010}

Treaty or not, the United States was desperate for intelligence about the Soviet missile program, and aerial photoreconnaisance was crucial to gaining it. The West's human intelligence capabilities within the Soviet Union were hamstrung by the intense surveillance under which their diplomats were placed, and the enormous size of the Russia itself.\footcite[p.~23]{lindgren_trust_2000} As the US relied increasingly on SIGINT to gather intelligence, the USSR began pursuing a more aggressive aerial defense policy, in which they demonstrated their willingness to open fire on foreign reconnaissance aircraft. The increased cost associated with these missions did little to deter senior US policymakers from attempting to surveil the skies.\footcite[p.~4]{pedlow_cia_1998}

% The Soviet Union was able to successfully destroy US Military aircraft during purported peacetime because in many cases those US aircraft were not supposed to be there in the first place. In many cases their flight path was a clear violation of Soviet territorial boundaries---a fact of which the administration was aware.\footcite{goodpaster_memorandum_1956}

In the rest of this chapter, I will argue that Eisenhower satisfied the need for intelligence that an Open Skies treaty would have provided by normalizing the collection of aerial intelligence and minimizing its diplomatic consequences. This process happened in two parts: first, through the routine resolution of military reconnaissance shootdowns, and second, by divorcing espionage from the military---symbolically---with the creation of a spy plane. For both the military and civilian missions, I will demonstrate how the even the most sensationalized incidents resulted in no more than an ineffectual diplomatic protest.

\section{Reconnaissance Flights}
\subsection{Timeline of early Cold War SIGINT missions}
Well before the U-2 Incident in 1960, the United States had been routinely flying military aircraft near Soviet airspace, and the USSR had been routinely shooting them down. For a conflict defined by its lack of direct skirmishes between the two superpowers, a surprising number of Americans died in these incidents, which are mostly forgotten to history. A Smithsonian investigation in 2017 found that, over the course of the Cold War, over 200 American pilots were lost to Soviet shootdowns---126 of which remain unaccounted for to this day.\footcite{glenshaw_secret_2017}

The purpose of these missions was to gather SIGINT (signals intelligence) identifying the location of critical radar installations along the Soviet border. The Air Force called flights in this style ``ferrets,''  where converted bombers were outfitted with advanced radio equipment and sent to intercept as much information about Soviet radio transmissions as possible.\footcite[p.~4]{peterson_maybe_1993} According to the CIA, these missions began as early as 1947.\footcite[p.~4]{peterson_maybe_1993} Some of these flights only flew near the border, remaining within traditionally recognized international waters. Others were ``overflights,'' missions which deliberately violated Soviet airspace to collect intelligence, at a much higher risk of failure. The earliest postwar overflight of the Soviet Union of which I can find definitive record is on May 10, 1949, when two RF-80As covered the Kurile islands\footcite[p.~8]{peebles_shadow_2000}

Determining the quantity of these flights is difficult, because they were a closely guarded secret at the time and information about them remains variously classified. What we do have is a public record of the flights that received media and diplomatic attention when they were shot down. The original stories are available in the newspaper archives, and the incidents have been helpfully compiled by US Air Force Academy historian Dr. John Farquhar, who specializes in aerial reconnaissance and wrote his master's thesis on how it shaped early Cold War diplomacy. Much of the research in this section, especially the analysis of which incidents attracted media attention, is based on his earlier work.
\newline

\begin{table}[ht]
\begin{tabular}{llr}
\textbf{Date}     & \textbf{\makecell[l]{U.S. Service \&\\ Aircraft Type}}   & \textbf{General Location} \\
8 April 1950      & USN PB4Y2 Privateer           & Baltic Sea                           \\
6 November 1951   & USN P2V Neptune               & Sea of Japan                         \\
13 June 1952      & USAF RB-29                    & Sea of Japan                         \\
7 October 1952    & USAF RB-29                    & East of Hokkaido/Kuril Is.           \\
29 July 1953      & USAF RB-50                    & Sea of Japan                         \\
4 September 1954  & USN P2V Neptune               & Sea of Japan                         \\
7 November 1954   & USAD RB-29                    & East of Hokkaido.Kuril Is.           \\
18 April 1955     & USAF RB-47                    & Off Kamchatka Peninsula              \\
10 September 1956 & USAF RB-50                    & Sea of Japan                         \\
2 September 1958  & USADF C-130                   & Soviet Armenia (near Turkish border) \\
1 May 1960        & CIA U-2                       & Sverdolsk, USSR                      \\
1 July 1960       & USAF RB-47                    & Barents Sea                          \\
10 March 1964     & USAF RB-66                    & East Germany
\end{tabular}
\caption{Summary of Soviet Shootdowns, 1950-1964}
\label{soviet-shootdowns}
\end{table}

In addition, the CIA history magazine \emph{Cryptologic Quarterly} analyzed the Soviet shootdowns of US reconnaissance aircraft in depth, creating a comprehensive (as far as we know) list of each American overflight that was shot down by the USSR. (Table \ref{soviet-shootdowns}) The article was written and distributed internally in 1993, but not declassified until 2009. The author, Michael Peterson, limits the table using two criteria the perfectly suit our purposes: the only flights listed are those that were (a) shot down down by Soviet attacks, not Chinese, North Korean, etc., and (b) explicitly US reconnaissance aircraft.\footcite[p.~4]{peterson_maybe_1993} Note that not all of these are necessarily overflights; many were in international waters when they were shot down. Reproduced below is a table of the 13 times that the USSR shot down an American aircraft designed for collecting intelligence between 1950 and 1964---intercepted espionage attempts.\footcite[p.~5. In the original document, this table lists the first incident as having taken place over the Barents Sea, not the Baltic Sea. Because the description of the mission---including a map of its route in the same document---takes place entirely over the Baltic sea, I have concluded that this has to be a typographical error, and corrected it here.]{peterson_maybe_1993}

\subsection{The Baltic Incident (April 8, 1950)}

The first of these shootdowns, on April 1950, brought the previously-hidden aerial reconnaissance program to the attention of the American people. An unarmed Navy Patrol plane, a PB4Y2 Privateer, was intercepted by Russians over the Baltic sea. The ambassador lodged a formal note of protest, incorrectly alleging that the US aircraft they had shot down was a B-29 bomber.\footcite{kirk_ambassador_1950} Upon receiving the note, Secretary of State Dean Acheson remarked that Ambassador Vishinsky's manner was ``serious but not aggressive or antagonistic,'' and the Secretary recommended that ``recommended that publicity our side should be avoided or, if unavoidable, minimized.''\footcite{kirk_ambassador_1950}

A few days later, the American ambassador issued his formal response, in which he correctly identifies the model of plane and claims that at no point did it cross into any territorial waters. He then puts the blame directly on the Soviets, and outlines the consequences that the Americans plan to impose, excerpted below:\footcite{the_associated_press_text_1950}

\begin{quote}
The United States Government demands that the the Soviet Government institute a prompt and thorough investigation of this matter in order that the facts set forth above my be confirmed to its satisfaction. The United States Government further demands that the most strict and categorical instructions be issued to the Soviet air force that there be no repetition, under whatever pretext, of incidents of this kind which are so clearly calculated to magnify the difficulties of maintaining peaceful and correct international relationships.

The United States Government fully expects that, when its investigation is completed, the Soviet Government will express its regret for the unlawful and provocative behavior of its aviators, will see to it that those responsible for this action are promptly and severely punished and will, in accordance with established custom among peace-loving nations, pay appropriate indemnity for the unprovoked destruction of American lives and property.
\end{quote}

This is a weak response. It imposes no consequences, threatens no sanctions, and demands only the most \emph{pro forma} of reparations. In essence, the note demands an apology and a promise not to repeat. There was absolutely no chance that the parties responsible on the Soviet side would be held accountable---in the most serious of cases, some lower-level officers would take the fall, and even that was unlikely to happen here. If the United States had followed through on its demand for reparations---the only material repercussion mentioned---then there is an argument to be made that this statement has some teeth. Instead, the demands were rejected in totality by the USSR.\footcite{salisbury_kremlin_1950} As far as I can tell, no reparations were ever made and no further consequences were imposed as a result.

Both sides had decent cause to make this a larger political issue. The Soviet Union had fired first on a US Navy aircraft---an aircraft which could not have been the aggressor because its only armament was a .45-cal pistol carried by one of the crewmen.\footcite[p.~7]{peterson_maybe_1993} And given that the plane was never found, all available evidence that we have today suggests that the flight likely did not violate the traditional 12-mile airspace boundary. Based on the last-received position report, the CIA claims that the plane was flying 20-25 miles off the coast of Latvia at the time it was shot down.\footcite[p.~7]{peterson_maybe_1993} Unnamed Navy and Air Force officers, speaking to \emph{The New York Times} in response to the Soviet protest note, conceded that it was possible that a complete breakdown in the plane's navigation system might have caused it to veer off-course into Latvia, but that based on its last known position, it would have taken ``a navigational error of nearly 90 degreees to cause the craft accidentally to wander over the Baltic states.''\footcite[Technically, the  article credits the first statement about the possible electronic breakdown to ``Navy and Air Force officials'' and the 90 degrees statement to ``Aerial navigators.'' It is unclear whether the navigators in questions are with the US military, but it does not matter regardless.]{the_new_york_times_soviet_1950}

Meanwhile, The United States knew that they hadn't lost a B-29, and it probably hadn't ended up in Soviet airspace---but the Privateer in question had been performing a SIGINT reconnaissance mission when it was shot down. Though the US was likely correct that it never flew over Latvia, it certainly got as close as it needed to in order to gather the necessary intelligence about Soviet military installations on the coast. No traditional boundaries were violated, but a Navy privateer loaded down with electronic reconnaissance equipment flying close to Latvia looked, to use a less technical term, shady. And still, based on the initial exchange of diplomatic protests, neither side was inclined to magnify the issue if they could avoid it.

The American press had no such reservations. \emph{Washington Post} columnist Walter Lippmann wrote that it could not have been ``a local incident brought on by a local commander but [rather] that it was an act of Soviet policy. The known facts indicate that Soviet intelligence \textelp{} believed it was carrying important electronic equipment and that orders were given to the Soviet fighter command to intercept it.''\footcite{lippmann_baltic_1950} He speculates that there is no way the plane could have violated Soviet airspace, because wreckage would have been recovered, the plane was too slow, and no commander would have sent it on such a dangerous mission intentionally. ``If, when, and as the American command were reconnoitering the Soviet military establishments on the Baltic coast,'' Lippmann wrote, ``it would use a plane of a wholly different type.''\footcite{lippmann_baltic_1950}

A newspaper columnist is not a policymaker, but his implicit argument is reflective of the general attitude towards espionage at the time, and that attitude is striking. Of course we do these things, we just do them \emph{better} than that---and if we had been doing the thing that the Soviets claimed we were, then it would have been justified of them to shoot us down. ``Baltic Plane Mystery,'' a \emph{Post} article written a few days later, is almost sympathetic to the Soviet commanders. ``Electronics make the old delimitations for border coastal flights ridiculous. A plane flying on a course perfectly legal by standards accepted today might still be engaged in reconnaissance of the first importance that an unfriendly power would try to frustrate.''\footcite{childs_baltic_1950} This is true, both in terms of the mission itself and the technological environment, but he goes even further. ``The only sound and safe assumption is that the Russians have a thorough and far-reaching espionage system. And at the same time we must hope that our system, and particularly on the side of counterespionage, is effective.'' Even the press, outraged at the loss of American life, seems to take it as a given that this kind of espionage is necessary, and expected from both sides.

The most important consequence of the Baltic Incident was not even diplomatic. As a direct result of the shootdown, President Truman, apparently largely uninformed about these missions, ordered a thirty-day halt to all reconnaissance flights. In that time, his office could come up with an operating procedure that would balance the strategic need for reconnaissance with its potential diplomatic volatility.\footcite[p.~41]{farquhar_aerial_2015} The resulting memo from the Joint Chiefs of Staff created, with Truman's approval, the Special Electronic Airborne Search Project (SESP), which set clear delineations of responsibility between Navy and Air Force reconnaissance, and established executive responsibility for approving the missions.

Dr. Farquhar cites this as a landmark moment in the history of aerial reconnaissance. ``No longer would ferret operations be conducted ad hoc by the military services; from 1950 onward'' he writes, ''reconnaissance operations attracted Presidential attention and played a significant role in shaping U.S. foreign policy.''\footcite[p.~42]{farquhar_aerial_2015} The strict procedures that SESP establishes are crafted to avoid the risk of escalation, while still explicitly acknowledging the ``serious disadvantages accruing to the United States if the cessation of these operations were to be extended over an excessively long period.''\footcite{bradley_memorandum_1950} But the SESP procedures also contain a curious provision: ``Flights will not be made closer than twenty miles to the USSR or USSR- or satellite-controlled territory.''\footcite{bradley_memorandum_1950} This would not be followed.

\subsection{Routine resolution}
Between the shootdown of the Navy Privateer in 1950 and the famous U-2 shootdown in 1960, the United States lost nine more aircraft to Soviet air defenses. While I do not have the space to analyze each one in the same level of detail, I can demonstrate how the general attitudes displayed in the Baltic Incident become the normalized attitudes towards peacetime aerial reconnaissance. In each of the incidents listed, the pattern is the same---each side releases dueling statements of protest, and situation continues more or less as before.

The work of cataloging diplomatic reactions to Soviet shootdowns in the 1950s has already been done by the aforementioned Dr. Farquhar, especially in his article for \emph{Air Power History}: ``Aerial Reconnaissance, the Press, and American Foreign Policy, 1950-1954.'' In it, he examines how each major shootdown was covered in the press, and how it impacted the relationship between the US and the Soviet Union. This list provides a valuable cross-reference with Peterson's table of reconnaissance shootdowns---for each incident that Dr. Farquhar mentions, we can check with the CIA to verify whether or not it was a reconnaissance mission.

Very briefly, here are the next few peacetime reconnaissance shoowdowns whose diplomatic consequences Dr. Farquhar analyzes. A B-29 that went missing on October 7, 1952 hit the front page of the New York Times two days later. This mission is on Peterson's list, and it was a ferret flight. Farquhar notes that the incident happened during US election season and that the media thought the attack was intended to lower American prestige, but he does not mention any diplomatic consequences.\footcite[p.~43-44]{farquhar_aerial_2015} I did find the official US response, however. Just like with the Baltic Incident, they simply demanded that the Soviets pay for the plane and return any survivors.\footcite{the_new_york_times_u.s._1952} Two more shootdowns in March 1953---one involving and American F–84 Thunderjet and the other an RAF Lincoln bomber---actually resulted in conciliatory tones from both sides, including apparently a secret meeting to reduce aerial tensions ``of which little is known''.\footcite[p.~45]{farquhar_aerial_2015}

In another incident on July 29, 1953, a B-50 (certified reconnaissance flight) was shot down over the Sea of Japan. The United States protested, and the Soviets counter-protested an alleged American shootdown of a Soviet passenger plane. Despite press outrage, no further action was taken.\footcite[p.~47]{farquhar_aerial_2015} A September 4, 1954 shootdown of a Navy P2V Neptune actually got a Republican senator to call for breaking off diplomatic ties with the Soviet Union, decrying the continued use of diplomatic notes that had clearly proved ineffective. Instead Eisenhower brought it to the UN, with the full knowledge that an unfavorable decision would be vetoed by the Soviets anyhow. Although the move was entirely symbolic, it signaled Eisenhower's ability to resist the calls for harsher action while still placating the American public.\footcite[p.~47]{farquhar_aerial_2015}

I do not wish to belabor the point, so suffice to say that Farquhar's conclusion is that after this moment in 1954, the resolution of these incidents became less tense. ``International incidents posed by shoot downs of reconnaissance aircraft still acted as a barrier in the path of détente; but, by late 1954, overriding strategic concerns dictated a move toward breaking the cycle of hostility.''\footcite[p.~49]{farquhar_aerial_2015} In terms of media coverage, and diplomatic significance, no shootdown that followed would require more significant consequences than the ones that had already happened.

There is a single incident from 1950-1954 that does not line up with the pattern. On November 20, 1951, an American C-47 transport that was forced down in Hungary. This incident is omitted from Peterson's list because, despite Soviet cries of ``spies and saboteurs,'' it wasn't actually a spy flight.\footnote{It also wasn't exactly shot down so much as forced to land.} Of all the cases that Farquhar examines, this is the only one that provokes a sustained diplomatic response---and it is the only one that definitively carried no espionage equipment. While the ferret flights were perfectly calibrated to fit as much radio surveillance equipment as possible, the only evidence of espionage that the Soviets were able to produce from the C-47---a plane that they recovered intact---was a portable radio, two extra parachutes, and some packets of warm blankets.\footcite{the_united_press_soviet_1951}

I contend that the lack of espionage equipment is precisely the reason the United States made a sustained public effort to recover the four airmen, which is something that almost never happened with these flights. Farquhar summarizes the American answer to this incident like this: ``Responding to the press attention, the Truman Administration acted swiftly, attempting to gain the fliers release through diplomatic pressure. The President ordered the Hungarian consulates in New York and Cleveland closed and banned private travel to the country. Legislatively, Truman asked Congress to pass a \$100 million Mutual Security Act to aid 'selected persons residing in Soviet bloc states or refugees who wanted to form armed units' in opposition to Communism.''\footcite[p.~43]{farquhar_aerial_2015} Truman immediately impacted Hungarian interests in the United States and threatened to empower \emph{armed dissidents} if the prisoners were not released. The contrast between this and just one year prior, when the US demanded ``a prompt and thorough investigation'' of the Baltic Incident, is stark.

We also know that the United States conducted many more overflights than the failed ones that are listed here, using a wide variety of retrofitted planes. In 2001, the Air Force held a symposium honoring and recollecting the pilots who flew the overflights on Soviet territory in the early Cold War, an effort in which they were joined by the British in 1954.\footcite[p.~v]{hall_early_2003} Many of the speakers represent entire reconnaissance projects that apparently didn't even register diplomatically. Project Heart Throb, which outfitted RB-57s with surveillance equipment, flew anywhere from 15 to 19 missions over Eastern Europe from 1955 to 1956, in the recollection of Gerald E. Cooke, a Air Force pilot assigned to the project.\footcite[p.~194]{hall_early_2003} Another project from around the same time, Slick Chick, set up F-100 jets with rapid-fire 20mm cameras. The pilot who presented about this project at the symposium, Cecil Rigby, personally flew two such missions. Though he was attempting to recall highly classified events from 50 years prior, he estimated that there were six Slick Chick missions total.\footcite[p.~176]{hall_early_2003}

The USSR obviously knew about these overflights, which were tracked by radar and often chased by MiG fighters. And unlike some of the flights that actually got shot down, these overflights were clear violation of their territorial sovereignty. We will likely never know exactly how many such missions were flown. The Truman and Eisenhower administrations were shockingly bad at recovering the pilots who were held as POWs by the Soviets. Dino Brugioni, a senior CIA officer involved in the aerial imagery program, paints a grim picture of the institutional accountability involved when the flights went bad. In meetings between the State department and high-level Soviet diplomats, the subject of captured pilots ``was broached only perfunctorily in relation to other things being discussed.''\footcite[p.~72]{brugioni_eyes_2010} Some family remembers of lost airmen received posthumous awards, but most simply received his personal effects and no explanation. ``And because these were secret missions conducted by a field command,'' he writes, ``a change of field commander meant that the fate of the men was soon forgotten.''\footcite[p.~72]{brugioni_eyes_2010} With missions so secret that we abandoned the men who flew them, we can no expectation of finding a complete record today.

Even without a complete list of flights, it is clear that the territorial boundaries of the USSR were routinely violated to generate critical intelligence for NATO powers in an early, relatively volatile stage of the Cold War. And as time went on these flights became more acceptable, not less. Diplomats of both nations were, as a matter of policy, able to paper over these issues in service of moving on to other matters. When the shootdown had unavoidable diplomatic consequences---when it was well-publicized---then they exchanged the necessary diplomatic cables and engaged in some saber rattling. These incidents undoubtedly altered the climate of Cold War diplomacy, but the fact remains that in 10 years of reconnaissance, the only thing the Soviets did to keep the United States from conducting these incredibly provocative reconnaissance flights was to shoot them down.

\section{From ferret flight to spy plane}
\subsection{Project AQUATONE}
While Eisenhower was able to defuse some of the tension around the military reconnaissance flights, even the most aggressive overflights could only hope to skirt the borders of Soviet territory. The United States couldn't just send a B-50 over Moscow. What US intelligence-gathering capabilities needed was some way of photographing suspicious installations deep within Russia while still being able to claim the same benefit-of-the-doubt status that prevented even the most contentious overflights from escalating into a larger conflict What they needed was a spy plane.

The key feature of a spy plane is that it is designed to be minimally provocative. As detailed earlier, any plane can be retrofitted with cameras or radios that allow it to perform useful reconnaissance functions. What differentiates a spy plane is that there can can be no doubt that it is being used for anything else. For the spy plane to serve its intended purpose as a minimally provocative agent of espionage then, two things must be true: its operation must be understood by adversaries to be non-military, and that distinction must meaningfully alter their threat perception and associated response. To build such a aircraft, Eisenhower approved Project AQUATONE, the code name for the the creation of the U-2.

From its conception, the U-2 was purpose-built to be coded as a civilian aircraft. Eisenhower required that the pilots of U-2 aircraft would be CIA employees, specifically forbidding uniformed Air Force officers.\footcite[p.~33]{lindgren_trust_2000} This is also the reason that that the CIA had operational control of the program over the Air Force. ``I want this whole thing to be a civilian operation,'' Eisenhower wrote to settle an operational dispute between the two departments. ``If uniformed personnel of the armed services of the United States fly over Russia, it is an at of war---legally---and I don't want any part of it.''\footcite[p.~60. The original source for this quote is an \emph{OSA History} that requires codeword clearance. It is quoted here by the History Staff of the CIA.]{pedlow_cia_1998} This also allowed the United States to ``truthfully deny,'' in the phrasing of the CIA, that any US ``military planes'' had flown over the USSR, in response to a Soviet protest note after U-2 overflights began in 1956.\footcite[p.~109]{pedlow_cia_1998}

This is not a media studies or philosophy thesis, but it is worth considering for a second how having the CIA run the program---ostensibly for the purpose of deception---changes the nature of the program itself. If the only goal here had been to achieve a level of plausible deniability in the the event of a shootdown, then not wearing an Air Force uniform during the mission seems like it would have been sufficient. The ``weather reconnaissance'' excuse had even been used before with the ferret flights.\footcite[p.~45]{farquhar_aerial_2015} At the highest levels of American government, policymakers didn't just want to pretend that AQUATONE was a civilian operation, they wanted it to \emph{be} a civilian operation---even if the planned NASA weather-craft cover didn't hold. ``It is of utmost importance to differentiate in our minds, and to cause the Russians to differentiate in theirs, between AQUATONE-type operations and reconnaissance by military aircraft'' reads a top-secret CIA memo from 1956.\footcite[p.~1]{miller_suggestions_1956} Both halves of that are equally important. The Russians must believe that these operations are peacetime civilian operations, and they must be correct about it.

The United States was aware how provocative these reconnaissance flights might be, and Eisenhower was always concerned, correctly, about ``the terrible propaganda impact that would be occasioned if a reconnaissance plane were to fail.''\footcite[p.~162]{pedlow_cia_1998} But he was also under tremendous political pressure to learn more about the Soviet missile program, as right-wing senators hyped up fears of a so-called ``missile gap,'' concerned that Soviet ballistic missile capabilities were advancing faster than our own.\footcite[Fears of this missile gap quickly followed earlier fears of a ``bomber gap,'' which ironically the U-2 had been critical in disproving.]{licklider_missile_1970} Starting in 1958, the American administration had ordered a drastic decrease in the number of overflights.\footcite[p.~51]{powers_operation_2004}

It turned out however, that the Soviets opted to lodge their diplomatic protests privately, and soon they stopped protesting them at all.\footcite[p.~42]{lindgren_trust_2000} The Eisenhower administration fatally misinterpreted this sign. ``It is one of the many unfortunate aspects of the high degree of secrecy surrounding Moscow's actions'', imagery analyst David Lindgren writes, ``that the United States never understood the how angered and frustrated Soviet leaders were made by the U-2 overflights.''\footcite[p.~52]{lindgren_trust_2000} Why were the Soviets so incensed, and what does that say the spy plane's ability to code as civilian and prevent conflict escalation? As with the military reconnaissance flights, that question can be answered by investigating the maximal case. Fortunately, there is a single moment that is unquestionably the most significant diplomatic drama to emerge from a Cold War spy plane mission: the shootdown of Gary Powers, best known as the ``U-2 Incident.''

\subsection{The U-2 Incident}
On the morning of May 1, 1960, Soviet Air Defense forces detected a high-altitude aircraft flying over Soviet Tajikistan. Though they were were not yet able to conclusively identify it, the foreign aircraft was an American U-2 spy plane performing an reconnaissance overflight of the USSR---at a time when the Soviets were deeply embarrassed by their inability to prevent such flights.\footcite{orlov_u-2_2007} A previous U-2 overflight on April 9, less than a month prior, had lasted a full six hours and the aftermath caused an upheaval in the Soviet military. Leadership ordered an investigative commission to root out the shortcomings of the Air Force and Air Defense agencies, and a series of charts were drawn up anticipating the routes of future U-2 flights. The next time one came through, the Soviets were absolutely determined to shoot it down.

Though better prepared, the Soviets still proved unable to immediately stop the flight on May \nth{1}. A missile battalion in the plane's path was not on alert duty. Armed fighter aircraft were in the wrong positions. A frantic attempt to have another fighter plane literally ram it out of the sky was scrapped when the pilot failed to make visual contact.\footcite{orlov_u-2_2007} The highest possible levels of Soviet leadership were actively involved with the mission as it was taking place. A Soviet Colonel in the USSR Air Defense later recalled that Khrushchev ``clearly viewed the violation of their nation's skies by a foreign reconnaissance aircraft on the day of a Soviet national holiday, and just two weeks before a summit conference in Paris, as a political provocation.'' \footcite{orlov_u-2_2007}

Why did the generally cautious Eisenhower authorize two overflights in such close proximity to a crucial summit? For one, the secrecy of the program worked against him. Eisenhower was not willing to make the U-2 program public, so he couldn't convince his critics that he had intelligence as precise as he did that no such missile gap existed.\footcite[p.~51]{lindgren_trust_2000} And the missions had been such a tremendous success that his reluctance to authorize more overflights grated his staff, especially the influential DCI Allen Dulles.\footcite[p.~354]{brugioni_eyes_2010}

Another crucial factor emboldened Eisenhower and his team---most of whom, like Dulles, were much more enthusiastic about overflights than he was---to conduct that second mission. The USSR had been aware of the first flight, and the United States \emph{knew} that they were aware of it thanks to an onboard computer which had detected the Soviet tracking from the beginnings of the operation.\footcite{pedlow_cia_1998} That the US took Soviet silence on this issue as encouraging was a severe miscalculation. After years of bluster about the superiority of Soviet military forces, Khrushchev was not about to admit that his Air Defense could not not shoot an unarmed spy plane flying deep into his homeland. They couldn't do anything about it, and they refused to admit that by lodging a protest.\footcite[p.~59]{powers_operation_2004} This was a military shame, and it was going to be resolved with a missile.

Almost four hours into the spy plane's flight on May 1, pilot Francis Gary Powers felt a dull ``thump'' and ``a tremendous orange flash lit the cockpit and sky'' as a Soviet surface-to-air missile exploded behind him, ripping the wings off his plane and sending it into a tailspin.\footcite[p.~61]{powers_operation_2004} His subsequent crash and capture by the Soviet military became an immediate international incident. Three days after NASA quietly reported that they had lost a U-2 type weather reconnaissance plane, Khrushchev announced that the USSR had shot down an American plane which had flown into their airspace, which he called ``an aggressive provocation aimed at wrecking the Summit Conference.''\footcite[p.~112]{powers_operation_2004} Once Khrushchev gleefully debunked the resulting American cover story by revealing that they had taken Powers alive, Eisenhower came clean in a Press Conference where he memorably declared that espionage was a ``a distasteful but vital necessity'' to prevent another attack like Pearl Harbor.\footcite{eisenhower_news_1960}

Publicly, the USSR was absolutely not ready to accept espionage as a ``distasteful but vital necessity.'' The Four Powers summit in Paris was disastrous. Khrushchev launched into an angry rant on the first day and stormed out, immediately squashing any hope that pressing issues like the fate of West Berlin would be resolved there. He also canceled a planned visit by Eisenhower to the USSR that was scheduled to take place the next month. A contemporary newsreel about the conference said that ``in the course of two hours, Khrushchev brought US-Soviet relations to their lower point since the end of World War II.''\footcite{universal_studios_summit_1960}

\subsection{The final normalization of spy flights}
No one disputes that the collapse of the Paris Summit is directly attributable to the Gary Powers shootdown, but that also appears to have been the only consequence. With the late rollout of the captured pilot, the dramatic scene at the summit, and canceling Eisenhower's visit, Khrushchev publicly embarrassed the United States without meaningfully altering Soviet policy towards them. In fact, the most lasting consequence of the U-2 Incident is that the two superpowers got better at resolving aerial disputes.

This is an unusual reading of the U-2 Incident, which is typically cited as having significantly worsened US-Soviet relations at time when it seemed like lasting peace might be possible. I don't intended to sound glib when I say that summits just aren't that important. From 1953 to the end of the Cold War in 1991, there were 23 total US-USSR summits.\footcite{fain_chronology_2011} The main issue that did not get resolved in 1960 was the status of Berlin. It also did not get resolved at the next summit in 1961, with Kennedy and Khrushchev. If the worst that happened because of the U-2 incident was an unproductive summit meeting, then that's a relatively minor punishment.

While the American strategy here is relatively clear and well-documented---collect high-risk intelligence sparingly and minimize the costs when caught---there is no better source for the Soviet strategy than Nikita Khrushchev's memoirs. The contrast between the severity of response Khrushchev wanted to portray and what the USSR actually did is highlighted by his own son Sergei's notes, who edited and annotated the 2007 edition which I am using.  On capturing Gary Powers, Khrushchev writes ``This was a hostile act by the leaders of the U.S. government, and they made no attempt to conceal it. They didn't think we had the capability of \textelp{} aquiriting irrefutable proof that the United States was using methods that were impermissible in peacetime.''\footcite[p.~239]{khrushchev_memoirs_2007}

That claim, that these were impermissible peacetime methods, is undercut by Sergei's footnote, pointing out what we already know, that more than forty American reconnaissance aircraft were shot down during the Cold War. Gary Powers was neither the first not the last pilot to surveil the USSR, but he was the only one of whom Khrushchev choose to make an example. If the methods were impressible, why wasn't a summit canceled with each flight?  Why weren't more of the over 200 American casualties---an unknown number of whom were captured---held up as trophies the way Khrushchev did Powers?

Two months to the day after Powers was shot down, July 1, 1960, a Soviet MiG-19 opened fire on an American RB-47H and captured the plane's navigator and the co-pilot. While Powers was serving a 10-year sentence in Soviet prison, the two airmen were returned to the US with moderate fanfare shortly after John F. Kennedy's inauguration, in early February, 1961. Khrushchev says in memoir that he wanted to continue ``our general line of peaceful coexistence'' and cites the resolution of this incident as an example of such, where the US was forced to make a formal request for the return of the airmen.\footcite[p.~256-257]{khrushchev_memoirs_2007} The only relevant concessions the US made in return was to announce the discontinuation of its U-2 overflights (which Kennedy was already committed to) and to not make an issue of their illegal detention.\footcite{time_cold_1961}

The best proof that aerial reconnaissance became just a blip on the diplomatic radar is Khrushchev's own summary of the impact of the U-2 incident. ``Now the commander of American forces in West Germany gave the order not to fly any closer than 50 kilometers from the border between East and West Germany. And no more incidents of that kind occurred.''\footcite[p.~256]{khrushchev_memoirs_2007} The footnote is brutal: ``In practice such incidents occurred again from 1961 onward, but instead of the U-2 the Americans now used various types of Phantom or SR-71.''\footcite[p.~258]{khrushchev_memoirs_2007} The US continued to conduct overflights with the purpose of surveilling USSR military installations, and on at least two occasions the plane crossed over into Soviet territory. These violations of their territorial sovereignty were a far cry from the invasive surveillance of previous missions, and the Soviets responded with... a diplomatic protest. The US quickly apologized, and that was that.\footcite{orlov_u-2_2007}



% We know that thinking on the reconnaissance flights shifted over the course of the Cold War, because despite the insistence of all parties to the contrary, the United States actually continued to use U-2 planes for reconnaissance missions!\footnote{The U-2 plane, a 60-year old aircraft equipped with a film camera, still flies missions today in Iraq and Afghanistan (\cite{phillips_u-2_2018}). The US has literally never stopped using it for intelligence purposes.} Officially, the CIA stopped sending missions into USSR airspace, and transitioned to he use of satellites for surveillance.\footcite{orlov_u-2_2007} Khrushchev claims that violations ceased after they shot down Powers, and credits himself with teaching the Americans that ``anyone who slapped us on our cheek would get his head kicked off.''\footcite[p.~449. An important note about Khrushchev's memoirs is that even though they were published after he was ousted, they were still written well before the end of the Cold War, smuggled to the West and published in part in 1970. While his writing is bombastic and of clear bias, it is certainly not pro-West, and there's no reason to suspect that he would be covering for the US here.]{khrushchev_khrushchev_1974} But the US continued to conduct overflights with the purpose of surveilling USSR military installations, and on at least two occasions the plan crossed over into Soviet territory. These violations of their territorial sovereignty were a far cry from the invasive surveillance of previous missions, and the Soviets responded with... a diplomatic protest. The US quickly apologized, and that was that.\footcite{orlov_u-2_2007} The US successfully normalized the use of U-2 planes for espionage purposes, within the confines of what the Soviets could tolerate, which was essentially that they weren't arrogant about it.


Over the course of decades, the USSR did not make a diplomatic issue of airborn reconnaissance. They simply took the necessary immediate countermeasures, and shot down the planes that they could. And the United States, in turn, does not appear to have made an enormous fuss when they were shot down. There are a number of reasonable explanations for why the U-2 shootdown was blown up beyond the traditional scope for reconnaissance flights, which again resulted in the loss of hundreds of Americans over the course of a few decades, none of whom received the attention that Powers did. Partially the Soviets felt humiliated by a technology for which they had no countermeasures, and the uniquely insulting way the overflights flew deep into Soviet territory. Some in the Eisenhower administration believed that it was an excuse to justify a turn towards an intensification of hostilities that had already been decided by hard-liners in the Kremlin.\footcite[p.~328]{kistiakowsky_scientist_1976}

\section{Conclusion}
Most scholars who study this period will argue that the cumulative tension created by these flights contributed to an overall chilly tone in US-Soviet relations. Dr. Farquhar, for instance, calls this the ``cycle of hostility,'' in which the series of aerial incidents increased suspicions of military buildup on both sides, leading to even more aerial incidents as both powers attempted to verify their suspicions.\footcite[p.~43]{farquhar_aerial_2015} This is definitely true, and yet it is not at all clear how this impacted any other aspect of the US-Soviet relationship. I can't definitively say that there isn't some diplomatic issue that would have gotten resolved at a summit absent these reconnaissance flights, but I can't find an obvious one either.

Consider the period of time over which these flights took place. If aerial reconnaissance had been anything more than a lingering issue, then it would have to have been resolved in some way before 1960. The US-Soviet relationship saw huge shifts, including periods of optimism, over the 11 years this was going on until the U-2 incident. Hundreds of flights taking place in contentious territory, many of them purposefully violating Soviet boundaries, some of them shot down, and the worst that happened in that whole time was a bad summit meeting, after which the flights continued on almost as before. Even with soldiers' lives on the line, both sides were so willing to contain the issue that many of the airmen never saw their families again.

A world where the United States did not conduct these flights would likely have been even more volatile. All accounts agree that Khrushchev succeeded in his public play to convince the world that the Soviet missile capabilities were far more advanced than they actually were.\footnote{I have a citation for this, but I'm not next to the book right now} Eisenhower was under tremendous pressure from the public, Republicans in Congress, and even his own staff to increase the Defense budget. One shudders to imagine how he would have been forced to respond if he hadn't been sitting on detailed photoreconnaissance reports that showed the Soviets were developmentally far behind what their public statements suggested. Even with that high-quality intelligence, he still oversaw a significant escalation of the Cold War arms race.

In 1992, at the end of the Cold War, members of NATO and The Warsaw Pact signed the ``Treaty on Open Skies,'' which establishes guidelines for unarmed aerial surveillance flights in the spirit of open information and de-escalation.\footcite{organization_for_security_and_co-operation_in_europe_treaty_1992} This concept was first suggested by Eisenhower in 1955.\footcite{center_for_arms_control_and_non-proliferation_fact_2017} In effect, it adds new elements to a practice which had been going on illegally for years with little incident. With certain restrictions, the kind of flight that altered the course of the Cold War is now officially routine and permissible. And of course, there are spy satellites now, a subject which will be addressed in Chapter 4.

Aerial reconnaissance is but one form of espionage, and at this stage many of the policymakers didn't even use that term. Still, it is clear that these flights were performing espionage, even if the distinction between military reconnaissance and peacetime intelligence wasn't yet fully formed. And we can see that despite ample provocation, leaders (in this case the USSR) choose not to escalate incidents whose sole purpose was to gather information on military capabilities. In the next chapter I will look at the impact of HUMINT, human intelligence gained from sources and spies, which is closer to what people typically think of when they think of espionage. These spies will be gathering information about on a lot of the same subjects that early aerial photoreconnaissance missions targeted. And in February 1962, Gary Powers walked across a Bridge into West Berlin at the same time as Russian Colonel Rudolph Abel, an American prisoner. Powers' father had suggested to the State Department that they might be traded. A spy for a spy.\footcite[p.~239]{powers_operation_2004}


% What about if they'd used it more?
% What if they'd used bombers instead?





% Two problems: they're secret and it's not clear which norms are relevant.
% What vector are we looking to analyze?

% How does secrecy fit in?
% How does "operational shit" like no assasinations fit in

% \section{Espionage in International Law}
% What's missing with existing scholarship?


% Why is it neccesary to analyze espionage in the context of norms
% How does legal fit in?


\newpage
\printbibliography

\end{document}
